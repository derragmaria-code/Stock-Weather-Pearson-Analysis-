\documentclass[11pt,a4paper]{article}

% --- Packages ---
\usepackage[utf8]{inputenc}
\usepackage[T1]{fontenc}
\usepackage{geometry}
\geometry{margin=1in}
\usepackage{booktabs} % For professional tables
\usepackage{graphicx} % To include the chart
\usepackage{amsmath}  % For mathematical notation
\usepackage{titlesec} % For custom headings
\usepackage{xcolor}   % For colored text highlights

% --- Title Information ---
\title{Analysis of Moroccan Equity Market Volatility vs. Regional Precipitation Patterns \\ \large An Investigation into the 2024--2026 January Performance}
\author{Data Analysis Report}
\date{February 8, 2026}

\begin{document}

\maketitle

\section{Executive Summary}
This report evaluates the statistical relationship between average precipitation in the Casablanca-Rabat axis and the closing prices of major securities listed on the Moroccan stock exchange. The study reveals a Pearson correlation coefficient ($r$) of \textbf{0.2754}, suggesting a weak but non-negligible positive relationship.

\section{Methodology}
The analysis was conducted using daily precipitation records merged from two major urban centers. The correlation was calculated using the Pearson product-moment correlation formula:
\[ r = \frac{\sum (x_i - \bar{x})(y_i - \bar{y})}{\sqrt{\sum (x_i - \bar{x})^2 \sum (y_i - \bar{y})^2}} \]
where $x$ represents the stock price and $y$ represents the average daily rainfall for the month of January.

\section{Data Summary}

\begin{table}[h]
\centering
\caption{Rainfall and Market Overview (January 2024--2026)}
\vspace{2mm}
\begin{tabular}{@{}lccc@{}}
\toprule
\textbf{Metric} & \textbf{2024} & \textbf{2025} & \textbf{2026} \\ \midrule
Avg Rainfall (mm) & 1.06 & 1.05 & 8.10 \\
Rainy Days        & 9    & 8    & 22   \\
Sample Stock (Managem) & 1,800 MAD & 2,600 MAD & 8,050 MAD \\ \bottomrule
\end{tabular}
\end{table}

\section{Statistical Interpretation}
The resulting correlation of \textbf{0.2754} indicates that approximately \textbf{7.6\%} ($R^2$) of the variance in the observed stock prices can be associated with variations in January rainfall. 

\begin{itemize}
    \item \textbf{2024--2025:} Market growth occurred despite stagnant rainfall levels, suggesting sectoral independence.
    \item \textbf{2026 Outlier Effect:} The significant surge in rainfall coincided with extreme price appreciation in the mining sector (notably Managem), which acted as the primary driver for the positive correlation value.
\end{itemize}

\section{Conclusion}
While a correlation of 0.27 is statistically considered weak, it highlights a secondary economic "pulse" in the Moroccan market. We conclude that while rainfall is not a primary predictor of stock valuation, it serves as a macro-environmental catalyst that supports bullish sentiment in specific industrial sectors.

\end{document}